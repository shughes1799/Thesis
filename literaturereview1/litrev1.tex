\chapter{Analysis of previous work analysing the $\omega\pi\pi$ decay channel}

\section{List of Papers}

\section{Overview}

In the energy range available in the G11 dataset the decay channel open to a wide range of possible decay routes through 7 different intermediate meson states ending in $\omega\pi\pi$. However the properties of most of these mesons are poorly understood with great uncertainty in the mass, decay width and branching ratios for the particles decaying to $\omega\pi\pi$, shown in the table below.

\begin{center}
	\begin{tabular}{|l|c|c|r|}
		\hline
		Meson & Mass (MeV) & Branching Ratio & p(MeV/c)  \\ \hline
		$a_2(1320)$ & $1318.3 \pm 0.5$ & $(10.6 \pm 3.2)\%$  & 366\\
		$\omega(1420)$ & $1400-1450$ & seen & 444 \\
		$a_0(1450)$ & $1474 \pm 19$ & seen & 484 \\
		$\omega(1650)$ & $1670 \pm 30 $ & seen & 617 \\
		$\omega_3(1670)$ & $1667 \pm 4$ & seen & 615 \\
		$\pi_2(1670)*$ & $1672.2 \pm 3.0$ & $(2.7 \pm 1.1)\%$ & 304 \\
		$\Phi(1680)$ & $1680 \pm 20$ & not seen & 623 \\
		\hline
	\end{tabular}
\end{center}
*Going via $\omega\rho \rightarrow \omega\pi\pi$ 

Amongst these states only the a2(1320) is easily distinguishable from other potential decays routes with a relatively narrow decay width and a clear separation in mass from other decay paths. The two states around 1450 MeV and the 4 states around 1670 MeV are much more challenging to isolate with the possible exception of the $\pi_2(1670)$ as a cut can be placed on the mass of the $2\pi$ around the mass of $\rho$.

\subsection{Previous studies looking at decays of $a_2(1320)$ to $\omega\pi\pi$}
There is limited previous work considering the decay of an $a_2(1320)$ to $\omega\pi\pi$. According to the particles data group there are no studies able to suitably determine the mass of the particle or the width of the decay. However there are 4 papers which provide data for branching ratios of the decay with respect to the majority $3\pi$ mode, typically via the $\rho\pi$ channel. These papers and their results are reviewed in the following section.

All the papers date from 1973-74 and use data collected from hydrogen or deuterium bubble chambers. The studies mostly used pion beams although the data from Defoix et al. was taken from a proton anti-proton experiment. The number of $A_2(1320)$ observed varies between 60 and 279 giving pretty limited statistics on the decay channel and from these determine branching ratios of between 0.10$\pm$0.05 and 0.28$\pm$0.09, normally for $\Gamma(A_{2}\rightarrow\omega\pi\pi)/\Gamma(A_{2}\rightarrow\rho\pi)$.

Defoix et al.\cite{defoix1973evidence} considered data taken from the CERN 80 cm hydrogen bubble chamber exposed to a proton-antiproton beam of 715 MeV/c. The studied considered the decay channel $\bar{p}p\rightarrow\omega^0\pi^+\pi^+\pi^-\pi^-$. This channel results in a total of 9 different possible $\omega^0\pi^+\pi^-$ combinations of which at most only one would be from the reaction of interest. Fits a Gaussian and a polynomial background to the mass plots of each 3 pion combination attempting to estimate the true number of omega events but simply combines all combinations in the analysis plots. Finds a significant enhancement at 1315 MeV by fitting a Breit-Wigner to a simple production prediction. Interprets this structure as a $a_2(1320) \rightarrow \omega^0\pi^+\pi^-$ decay mode instead of the more common $a_2(1320) \rightarrow \rho^{\pm}\pi^{\pm}.$ The paper also demonstrates a strong correlation of this decay with a $\omega\pi$ intermediate channel around 1040 MeV. However these results are no longer taken as part of the determining datasets by pdg. More recent data indicates there is a b1(1235) decay channel which decays to $\omega\pi$ however this is at a significantly higher mass and is not a known decay mode of the $a_2(1320)$.

U. Karshon et al.\cite{karshon1974structure} present data from a 4.93 GeV/c $\pi^+$ beam taken utilising the 82-in hydrogen bubble chamber at the Stanford Linear Accelerator. An enhancement is observed around the $a_2(1320)$ mass for $\omega\pi^+\pi^-$ in the reaction $\pi^{+}p \rightarrow p\pi^+\pi^+\pi^-\pi^+\pi^-\pi^0$ where $\omega \rightarrow \pi^+\pi^-\pi^0$. This study does not attempt to distinguish directly the combination of pions that were from the omega decay. They simply define $\omega$ events with the cut $0.76 \le M(\pi^+\pi^-\pi^0) \le 0.81 GeV$. This could lead to multiple combinations per event and it is not clear whether they select the closest value to the known $\omega$ mass or just let all combinations that pass the cut through. For each event selection 4 different methods of fitting were applied to the data. This involved fitting the data with a relativistic Breit-Wigner for the signal and one of 3 different types of phase space background, or alternatively a hand drawn background shape. Branching ratios for the channel in comparison to the dominant $a_2(1320)$ decay mode of $\rho^{\pm}\pi^{\pm}$ were calculated for each fitting method, taking the mass of the $a_2$ to be 1335 MeV and the width to be 100 MeV. Averaging over these methods they obtained $\Gamma(A_{2}^{0}\rightarrow\omega\pi\pi)/\Gamma(A_{2}^{0}\rightarrow\rho\pi)=0.29\pm0.08$. The study also interprets results of the decay for a charged $\omega^{+}$ again using a nonrelativistic Breit-Wigner along with a hand drawn background to fit to the data. For these results a branching ratio of $\Gamma(A_{2}^{+}\rightarrow\omega\pi\pi)/\Gamma(A_{2}^{+}\rightarrow\rho\pi)=0.10\pm0.04$ is obtained. In the particle data group listing the value used is an average of the two results and taken to be $\Gamma(A_{2}\rightarrow\omega\pi\pi)/\Gamma(A_{2}\rightarrow\rho\pi)=0.18\pm0.08$.

V. Chaloupka et al.\cite{chaloupka1973measurement} studied several different branching ratios for decays from the $A_2$ meson including the $\omega\pi\pi$ mode of interest to this study. The data was collected using a 2m hydrogen bubble chamber at CERN that was exposed to a 3.93 GeV/c beam of negative pions, leading to the reaction $\pi^{-}p\rightarrow pA_{2}$. Fitting the $M(\pi^{+}\pi^{-}\pi^{-}+neutrals)$ spectrum with a relativistic Breit-Wignar and a third order polynomial, fixing the mass and width to be those determined by the $\rho\pi$ mode data ($1306\pm9$ and $99\pm15$ MeV resepectively). Placing cuts of missing mass squared $< 0.3 GeV^{2}$ and Mandelstam $t < 1.0 GeV^{2}$. A branching ratio $\Gamma(A_{2}\rightarrow\omega\pi\pi)/\Gamma(A_{2}\rightarrow\rho\pi)=0.10\pm0.05$ is determined based on a sample size of 273 events. This study does not discuss the combinatorial background inherent in this channel and discusses their method of selection for identifying $\eta$ events in another decay mode of the $A_2$ with a simple cut around the $\eta$ mass. As it is not specified differently it is likely a similar approach was taken to the $\omega$ decay mode. 

J. D\'{\i}az et al.\cite{diaz1974evidence} presented evidence for $\omega\pi\pi$ decay modes of $A_2$ and $\omega_{3}(1670)$. The study analysed data produced with a 6 GeV/c positive pion beam incident on a deuterium bubble chamber at the Argonne national laboratory. Considering the reaction $\pi^{+}d \rightarrow p_{s}n\pi^+\pi^-\pi^+\pi^-\pi^0$ where $p_s$ refers to the spectator proton. The study identified significant enhancements at 1.3 and 1.65 GeV in the $M(\omega\pi\pi)$ spectrum fitted with a combination of 2 Breit-Wigner functions and a modified phase space background. The cuts applied ensure limited influence of the spectator proton with a maximum momentum of $<$ 0.250 GeV/c and require a $3\pi$ mass combination $750\le M_{3\pi} < 825$ MeV near the $\omega$ mass. However the study does not try to resolve between the different pion combinations and simply chooses selections that fit in the necessary mass range. There is also no discussion of the additional potentially overlapping states close to the enhancement at around 1.65 GeV apart from their selection of what is now known as the $\omega_{3}(1670)$. The study analyses both the $3\pi$ and $5\pi$ decay channels from the $A_2$ and uses these to calculate a branching ratio of $\Gamma(A_{2}\rightarrow\omega\pi\pi)/\Gamma(A_{2}\rightarrow\rho\pi)=0.28\pm0.09$.

At the time of writing none of the analyses used by the PDG to provide values for the $a_{2}(1320)$ mass and decay width studied the $A_{2}\rightarrow\omega\pi\pi$ decay mode. The only studies highlighted provide evidence for the branching ratio of the particle to this decay state. Each of these studies were quite limited in their statistics and their methods for background subtraction and fitting. Mass spectrums were fitted by selecting either a Gaussian or a type of Breit-Wigner in combination with a polynomial or unspecified modified phase space background, or even in some cases a hand-drawn background. Little justification for these choices was provided beyond that adding a typical resonant structure such a Breit-Wigner improved the quality of the fit. Several mentioned but none properly addressed the combinatorial background inherent in this channel and made little attempt to remove any bias created by selecting combinations that fitted with expectations for the $\omega$ or $A_2$ mass. The results of these studies varied significantly in their conclusions with quite large errors arising from the methodology. Also of note is that none of these studies used a photon beam as was used in the g11 experiment at CLAS.







\section{additional useful bits}

\begin{center}
 \begin{tabular}{|l|r|}
 	\hline
 	Channel & Branching Ratio  \\ \hline
 	$\omega \rightarrow \pi^+\pi^-\pi^0$ & $89.2\% \pm 0.7\%$ \\
 	$\omega \rightarrow \pi^0\gamma$ & $8.28\% \pm 0.3\%$ \\
 	$\omega \rightarrow \pi^+\pi^-$ & $1.53\% \pm 0.1\%$ \\
 	\hline
 \end{tabular}
 \end{center}
 
 \begin{enumerate}
 	\item $\pi^{-}_{1}\pi^{+}_{1}$
 	\item $\pi^{-}_{1}\pi^{+}_{2}$
 	\item $\pi^{-}_{2}\pi^{+}_{1}$
 	\item $\pi^{-}_{2}\pi^{+}_{2}$
 \end{enumerate}
 
 