\chapter{Introduction}
A fast timing hodoscope has been developed as part of the new forward tagger to be installed as part of the 12 GeV upgrade to the Continuous Electron Beam Facility at the Thomas Jefferson National Laboratory.

Introductory section needs to illicite the key physics objectives of the program. Why these are significant and why this is the approach to determine more information about them.

Probably an introductory section focused on quantum chromo dynamics, the standard model and mesons. Providing a background for the purpose of the hodoscope and the analysis.
\section{Theory}
Probably a good idea to have a general chapter on quantum chromo dynamics, its importance and direction of research.

Quark model and its limitations. What it describes well but also what it doesn't account for. Expectations from current theory predict states beyond the simple bound states of meson and baryon.

Confinement
Complexity of the coupling constant in comparison to QED.

Why study mesons? Meson spectroscopy.

Meson Nonets produced by the possible quantum states.
Higher energy states of mesons.
Hybrid mesons, tetraquarks, pentaquarks, hexaquarks/dibarions, excited gluon couple to a meson. These states are predicted but not currently confirmed. Short lived states that allow states with normally forbidden combinations of quantum numbers.

Discussion on the observation of resonances and their place on riemann sheets in the complex plane. Resonances occuring at points of discontinuity when the function rises to infinity. Since my knowledge is limited about this area probably a brief overview would be enough.

Much of the progress in the theory in the area stalled in the 1970s limited by the experimental data available at the time. Technology has advanced in the intervening period. With new detector systems coming online, significant potential for breakthroughs in the area. Lattice QCD also providing a new avenue for progress with the growth in computation power available in parralelised cpu farms. These developments provide constraints to the theory and helping to guide the experimental search. 


Experimental facilities researching hadronic nuclear physics. Compass, JLAB, MAINZ, BES-III, CERN, SLAC etc. 

Theory and background papers
\cite{dudek2011lattice}
\cite{jaffe1977multiquark}
\cite{bali1993comprehensive}
\cite{johnson1975bag}
\cite{dombey1969scattering}
\cite{schilling1973analyse}